\documentclass{article}

\usepackage{graphicx}
\usepackage[dutch]{babel}
\usepackage{hyperref}
\usepackage{amsmath, amssymb, amsthm}
\usepackage{siunitx}
\usepackage{tabu}
\usepackage{booktabs}% for better rules in the table
\usepackage[left=2.5cm,top=2cm,right=2.5cm,bottom=2cm,a4paper]{geometry}
\date{Academiejaar 2020 -- 2021}
\title{Teamkalender- Team 6}
\author{Team 6}


\begin{document}
	
\section{Samenvatting klantenvereisten}
Budget:
\begin{itemize}
	\item 3500 eenheden
\end{itemize}

Wat moet het miniatuur robotwagentje kunnen:

\begin{itemize}
	\item straten (lengte 1m) volgen in miniatuur stad
	\begin{itemize}
		\item Straten bestaan uit strepen van plakband.
		\item plakband is helder en 25mm breed
		\item ondergrond donker
		\item met sensor de plakband volgen
		\item lijn in de richting van de wagen: rijden
		\item lijn dwars op de baan: stoplijn -> stoppen (breedte 50mm)
		\item voorgeprogrameerd traject
	\end{itemize}
	\item wagens (voorliggers) detecteren
	\begin{itemize}
		\item afstand inschatten
		\item zorgen dat het op tijd kan stoppen om botsing te vemijden
	\end{itemize}
	\item verkeerslichten interpreteren op een hoogte van 7.5 cm
	\begin{itemize}
		\item kleur kunnen detecteren rechts van de wagen
		\item groen: wagentje mag doorrijden
		\item rood: blijven scannen tot het groen wordt
		\item licht knippert aan 1Hz
	\end{itemize}
	\item het moet bestuurbaar zijn vanop afstand
	\begin{itemize}
		\item men moet vanaf een afstand het wagentje kunnen besturen
		\item tijdens het traject overgenomen kunnen worden
	\end{itemize}
	\item redelijke snelheid
	\begin{itemize}
		\item moet snel kunnen stoppen als de stopstreep gedetecteerd wordt (minder dan 10cm/s)
		\item moet geen uur doen over het traject (sneller dan 1cm/s)
	\end{itemize}

\end{itemize}
\end{document}
