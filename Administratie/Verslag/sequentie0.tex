\documentclass{article}

\usepackage[dutch]{babel}
\usepackage{pdfpages}
%\usepackage[latin1]{inputenc}
\usepackage{tikz}
\usetikzlibrary{shapes,arrows}


\begin{document}
	sequentie 0
	
	\tikzstyle{decision} = [diamond, draw, fill=orange!20, text width=4.5em, text badly centered, node distance=3cm, inner sep=0pt]
	\tikzstyle{block} = [rectangle, draw, fill=blue!20, text width=5em, text centered, rounded corners, minimum height=4em]
	\tikzstyle{line} = [draw, -latex']
	\tikzstyle{cloud} = [draw, ellipse,fill=red!20, node distance=3cm,
	minimum height=2em]
	\tikzstyle{trapezium} = [trapezium, draw, fill=green!20, trapezium left angle = 65,trapezium right angle = 115, trapezium stretches,minimum width=4cm,minimum height=2cm,text width=4.5em]
	
	\begin{tikzpicture}[node distance = 3cm, auto]\label{flowchart reflectiesensor}
		%place nodes
		\node [trapezium](0){Waarde 0};
		\node [block, below of = 0] (begin) {Gebruik data reflectiesensor};
		\node [decision, below of=begin] (lijn) {Lijn gedetecteerd?};
		\node [block, below right of=lijn] (hervat) {Stop en wacht};
		\node [decision, below left of=lijn] (welke) {Stoplijn?};
		\node [block, below left of=welke] (stoplijn) {Ga naar sequentie 1};
		\node [block, below right of=welke] (volglijn) {Volg de lijn};
		\node [block, below of=volglijn] (hervat2) {Blijf op sequentie 0};
		%place paths
		\path [line] (begin) -- (lijn);
		\path [line] (lijn) -- node {ja} (welke);
		\path [line] (lijn) -- node {nee} (hervat);
		\path [line] (welke) -- node {ja} (stoplijn);
		\path [line] (welke) -- node {nee} (volglijn);
		\path [line] (volglijn) -- (hervat2);
		\path [line] (hervat) |- (hervat2);
		
	\end{tikzpicture}
\end{document}