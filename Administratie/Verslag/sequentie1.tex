\documentclass{article}

\usepackage[dutch]{babel}
\usepackage{pdfpages}
%\usepackage[latin1]{inputenc}
\usepackage{tikz}
\usetikzlibrary{shapes,arrows}


\begin{document}
	sequentie 1
	
	\tikzstyle{decision} = [diamond, draw, fill=blue!20, text width=4.5em, text badly centered, node distance=3cm, inner sep=0pt]
	\tikzstyle{block} = [rectangle, draw, fill=blue!20, text width=5em, text centered, rounded corners, minimum height=4em]
	\tikzstyle{line} = [draw, -latex']
	\tikzstyle{cloud} = [draw, ellipse,fill=red!20, node distance=3cm,
	minimum height=2em]
	
	\begin{tikzpicture}[node distance = 3cm, auto]\label{flowchart kleursensor}
		%place nodes
		\node [decision] (Rood) {Was het rood?};
		\node [decision, below left of=Rood] (Groen) {Is het groen?};
		\node [decision, below right of=Rood] (controle) {Is het rood?};
		\node [block, below left of=Groen] (volgende) {Sequentie nr. 2};
		\node [block, below right of=Groen] (opnieuw) {Opnieuw sequentie nr. 1};
		\node [block, below of=controle] (waarde) {Onthou de waarde};
		\node [block, below of=waarde] (herbegin) {Sequentie nr. 1};
		%place paths
		\path [line] (Rood) -- node {ja} (Groen);
		\path [line] (Rood) -- node {nee} (controle);
		\path [line] (Groen) -- node {ja} (volgende);
		\path [line] (Groen) -- node {nee} (opnieuw);
		\path [line] (controle) -- (waarde);
		\path [line] (waarde) -- (herbegin);
		\path [line] (waarde) -- (Rood);
		
	\end{tikzpicture}
\end{document}