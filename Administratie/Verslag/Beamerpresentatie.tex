%% This classfile tries to implement the lay-out of the beamer style beamerthemekuleuven2. This .sty-file can be downloaded here: https://www.kuleuven.be/communicatie/marketing/templates/presentatiemateriaal/index.html Not all options of the style are implemented in this class, for its purpose is merely to mimic the lay-out, and to provide a way to change the lay-out of presentations that were made using the "old" kulakbeamer class. For new documents, we recommend using the .sty-file instead.

\documentclass
   [kulak] % options: kul or kulak (default), handout 
   {kulakbeamer}

\usepackage[dutch]{babel}
\usepackage[utf8]{inputenc}
\usepackage[T1]{fontenc}

\title[Korte titel]{Lange titel}
\subtitle{Ondertitel}
\author[Korte naam]{Lange naam} 
\institute[Kulak]{KU Leuven Kulak}
\date{Academiejaar 2020 -- 2021}

%% Overview at begin of each section; delete if unwanted.

\AtBeginSection[]{
	\begin{frame}
	\frametitle{Overzicht} %Change to "Outline" for English presentation
	{
		\hypersetup{hidelinks} %disable link colors
		\hfill	{\large\parbox{.95\textwidth}{\tableofcontents[currentsection,hideothersubsections]}}
	}
\end{frame}}

\begin{document}

\begin{titleframe}
\titlepage
\end{titleframe}

\begin{outlineframe}[Overzicht]
\tableofcontents
\end{outlineframe}

 % % % Here you go  % % % 

\section{Inleiding}

\begin{frame}
\frametitle{Inleiding}
\begin{itemize}
	\item Wie: 1$^{ste}$ jaar bachelorstudenten IW 
	\item Waarom: Probleem oplossen en ontwerpen
	\item Wat:
		\subitem Zelfrijdend autootje met principe smart city
	\begin{block}{Definitie: Smart City}
		Stad met informatietechnologie om de stad te beheren en te besturen \cite{SmartCity}
	\end{block}
\end{itemize}
\end{frame}


\section[Opdracht]{Opdracht}% eventueel aanpassen

\begin{frame}
\frametitle{De V's van vereisten}
\begin{itemize}
	\item Volglijnen, stoplijnen en verkeerslichten interpreteren
	\item Voorgaande wagen detecteren en botsing voorkomen
	\item Voldoende grote snelheid
	\item Vanop afstand kunnen ingrijpen
\end{itemize}

\end{frame}



\section[Aanpak]{Aanpak}

\begin{frame}
\frametitle{Aanpak}
Tekst.
\end{frame}



\section[Experimenten]{Experimenten met LabVIEW}

\begin{frame}
	\frametitle{Experimenten}
	\begin{itemize}
		\item Voorbeeldprogramma's voor sensoren
		\item Testen
		\subitem Reflectiesensor: witte bladzijde en zwart omhulsel laptop
		\subitem Afstandssensor: afstand object variëren
		\subitem nog aan te vullen
	\end{itemize}.
\end{frame}



\section{Besluit}
\begin{frame}
\frametitle{Besluit}
Afsluitende tekst.
\end{frame}

\begin{frame}
\frametitle{Bronvermelding}
	\bibliographystyle{plain}
	\bibliography{bronnen_verslag}
	\bibliographystyle{unsrt}
\end{frame}

\end{document}
