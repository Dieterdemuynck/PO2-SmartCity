\documentclass[a4paper,twoside,kulak]{kulakreport} %options: kul or kulak (default)

\usepackage[dutch]{babel}

\faculty{Wetenschap \& Technologie Kulak}
\group{Ingenieurswetenschappen}
\title{Smart city}
\subtitle{Tussentijds verslag}
\author{Groep 6}
\institute{KU Leuven Kulak, Wetenschap \& Technologie}
\date{Academiejaar 2020 -- 2021}
\address{
   KU Leuven Kulak           \\
   Wetenschap \& Technologie \\
   Etienne Sabbelaan 53, 8500 Kortrijk             \\
   Tel.\ +32 56 24 60 20     \\
  
   }

\begin{document} % hier begint de eigenlijke inhoud van het document
%Wat allemaal in het doc verwerkt moet zitten:
%-klantenvereisten
%-ontwerpsspecificaties
%-ontwerpskeuze
%-financieel rapprot
%-evaluatie wat gedaan is en wat nog moet gebeuren en hoe
%-wat nog verbeterd kan worden aan ontwerp

\titlepage

\tableofcontents

\chapter*{Inleiding}
%probleemstelling

%Ontwerpsproces en planning
\chapter{Eerste hoofdstuk}
\section{Sectie-titel}
\subsection*{Inleiding}

<<<<<<< Updated upstream
\section{Deel 1 ontwerpproces}
%Per onnderdeel verwijzen voor bibliografie!!!

Eerst en vooral is het belangrijk om een idee te krijgen over alle cruciale onderdelen die zullen gebruikt worden in het project. 

Een van de basisonderdelen is de chassis. Op dit onderdeel zal alles worden gemonteerd. Het is als het ware de ruggengraat van de auto.%wikipedia chassis
Opdat de auto kan rijden zijn uiteraard wielen nodig. In het model dat verder wordt beschreven, worden er twee reguliere ronde wielen gebruikt en één bal caster. Om de aandrijving van de wielen mogelijk te maken wordt er aan ieder rond wiel een micro tandwielmotor geplaatst. De regeling van de motoren gebeurt via de dual drive motor. Om uiteindelijk geheel de auto te laten voortbewegen is er een nood aan een hardwarecomponent namelijk een microcontroller. Deze fungeert als een soort mini computer. Daarnaast zal ook extra stroomtoevoer moeten worden voorzien. Hiervoor worden twee oplaadbare lithium-ion batterijen gebruikt. Verder is het ook nuttig om gebruik te maken van een breadboard. Dit is echter geen onderdeel van de zelfrijdende auto, maar is handig voor het testen van elektrische circuits.


Wat auto moet kunnen en algemeen wat je zal gebruiken.
- herkennen lijnen: welke soort sensor (bv reflectie voor lijnen)
- stoplicht: kleurensensor, tot stilstand komen blabla
- snelheid: massa belangrijk
- andere wagens detecteren => remafstand, massa belangrijk => stoppen door signaal zodat motoren vertragen
- op afstand ingrijpen 

\subsection{Deel 2 ontwerpproces}
%Zeer specifiek toelichten waarom je elk onderdeel heb gekozen.

In deze sectie wordt aanvullende uitleg gegeven over elk onderdeel dat in het project wordt gebruikt. Daarnaast wordt extra beargumenteerd waarom dit specifiek onderdeel wordt gekozen.

Als eerste wordt de chassis gesproken. Voor deze auto wordt een rechthoekige chassis gebruikt met afmetingen 80 mm op 172 zoals kan gevonden worden op site %verwijzing chassis materiaal
Deze chassis is zeer handig in gebruik wegens de verscheidene groottes van de groeven. Bovendien is de rechthoekige vorm zeer gemakkelijk om alle componenten van de auto vast te hechten. Een ronde chassis zoals bijvoorbeeld van de site %site van de ronde chassis
is hiervoor minder geschikt. Bovendien zijn er in deze laatste groeven aanwezig voor de wielen. Dit impliceert dat er minder ruimte is om andere onderdelen te assembleren op het onderstel. 






=======
- reflectie sensor: wrm analoge en niet digitale
\section{Planning}
%Eerst hebben we nagedacht over welke materialen we nodig zullen hebben. Daarvoor is het best om vooral veel opzoekingswerk te doen om meer te weten te komen over de materialen. Het was vooral belangrijk om na te gaan welke onderdelen samengaan en welke niet. Dan wogen we af welke opties we hebben. Voor de aankoop van de materialen maakten we twee alternatieve modellen.
Eerst bestelden we onze materialen. Van deze hebben we CAD-modellen en technische tekeningen gemaakt om inzicht te krijgen in de opbouw. We zijn voorlopig nog bezig met de implementatie en de assemblage van het wagentje.

%Hoe gaan we dat doen




>>>>>>> AdministratieveDocumenten
\chapter*{Besluit}
Afsluitende tekst.



\bibliographystyle{plain}
\bibliography{bronnen_verslag}
\end{document}
