\documentclass[a4paper,twoside,kulak]{kulakreport} %options: kul or kulak (default)

\usepackage[dutch]{babel}

\faculty{Wetenschap \& Technologie Kulak}
\group{Ingenieurswetenschappen}
\title{Smart city}
\subtitle{Tussentijds verslag}
\author{Groep 6}
\institute{KU Leuven Kulak, Wetenschap \& Technologie}
\date{Academiejaar 2020 -- 2021}
\address{
   KU Leuven Kulak           \\
   Wetenschap \& Technologie \\
   Etienne Sabbelaan 53, 8500 Kortrijk             \\
   Tel.\ +32 56 24 60 20     \\
  
   }

\begin{document} % hier begint de eigenlijke inhoud van het document

\titlepage

\tableofcontents

\section*{Inleiding}
%probleemstelling

%Ontwerpsproces en planning
\section{Sectie-titel}

\subsection{Deel 1 ontwerpsproces}


Wat auto moet kunnen en algemeen wat je zal gebruiken.
- herkennen lijnen: welke soort sensor (bv reflectie voor lijnen)
- stoplicht: kleurensensor, tot stilstand komen blabla
- snelheid: massa belangrijk
- andere wagens detecteren => remafstand, massa belangrijk => stoppen door signaal zodat motoren vertragen
- op afstand ingrijpen

\subsection{Deel 2 ontwerpsproces}
Zeer specifiek toelichten waarom je elk onderdeel heb gekozen.

- reflectie sensor: wrm analoge en niet digitale



\section{Planning}

\chapter*{Besluit}
Afsluitende tekst.

\end{document}
