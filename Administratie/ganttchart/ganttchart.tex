\documentclass[dutch,landscape,12pt]{article}
% style
\usepackage{babel}
\usepackage{translator}
\usepackage[left=2.5cm,top=2cm,right=2.5cm,bottom=2cm,a4paper]{geometry}
\usepackage{fancyhdr}
\pagestyle{fancy}
\lhead{Team 6}
\rhead{Gantt chart}
\renewcommand{\headrulewidth}{0.4pt}
\usepackage{color}

% gantt
\usepackage{pgfgantt}
\def\pgfcalendarweekdayletter#1{%
	\ifcase#1M\or D\or W\or D\or V\or Z\or Z\fi%
}
\def\pgfcalendarmonthshortname#1{ 
	\translate{
		\ifcase#1\or Jan\or Feb\or Maa\or Apr\or Mei\or Jun\or Jul\or Aug\or Sep\or Okt\or Nov\or Dec\fi}
}


\definecolor{foobarblue}{RGB}{0,153,255}
\definecolor{foobaryellow}{RGB}{234,187,0}
\definecolor{grey}{RGB}{170,170,170}

\newganttchartelement{foobar}{
	foobar/.style={
		shape= rectangle,
		inner sep=0pt,
		draw=grey!70!blue,
		thick,
		fill=white,
		inner color=blue!10, outer color=blue!40, opacity=0.95
	},
	foobar incomplete/.style={
		/pgfgantt/foobar,
		draw=foobaryellow,
		bottom color=foobaryellow!50
	},
	foobar label font=\slshape,
	foobar left shift=0,%.1,
	foobar right shift=0%-.1
}

\begin{document}

% Een goede Gantt-chart maakt gebruik van \emph{links} en \emph{milestones}. Deze laatste definieer je in een lijstje onder de Gantt chart.

	\begin{figure}
		\centering
		\begin{ganttchart}[hgrid, vgrid, x unit=5mm, time slot format=isodate]{2021-02-12}{2021-04-02}
			\gantttitlecalendar{week,month=shortname,day,weekday=letter} \\
			
			\ganttgroup{1}{2021-02-12}{2021-02-16} \\

			\ganttgroup{2}{2021-02-26}{2021-03-26} \\ % {nummer}{begindatum}{einddatum}
			%2.1 Stuklijst
			\ganttfoobar[name=21]{2.1}{2021-02-26}{2021-02-26} \\
			%2.2 Assemblage
			\ganttfoobar[name=25]{2.2}{2021-03-19}{2021-03-19}
			\ganttfoobar[name=25]{2.2}{2021-03-26}{2021-03-26} \\ %Asssemblage: eind
			
			%3 3D modellen
			\ganttgroup[name=22]{3}{2021-02-26}{2021-03-12} \\
			\ganttlink{21}{22} % 21 must be completed before 22 %pijltje
			%4 Technische tekeningen
			\ganttgroup[name=23]{4}{2021-03-05}{2021-03-19} \\
			\ganttlink{22}{23}
			%5 Implementatie
			\ganttgroup[name=24]{5}{2021-03-12}{2021-04-02} \\
			
		 maart
			
		
			\ganttgroup{6}{2021-03-12}{2021-04-02} \\
			%3.1 Tussentijds verslag
			\ganttfoobar[name=31]{6.1}{2021-03-12}{2021-04-02} \\
		
			%3.2 Verslag afwerken
			\ganttfoobar[name=32]{6.2}{2021-04-23}{2021-05-21} \\
			%\ganttlink{31}{32}
			
			%3.3 Tussentijdse presentatie
			\ganttfoobar[name=33]{6.3}{2021-03-19}{2021-04-02} \\
		
			%3.4 Presentatie afwerken
			\ganttfoobar[name=34]{6.4}{2021-04-23}{2021-05-21} \\
			%\ganttlink{33}{34}
			
			
			%\ganttmilestone{Mijlpaal I}{2021-03-05} % milestone completed
			
		\end{ganttchart}
	\end{figure}
\newpage

\begin{figure}

	\centering
	\begin{ganttchart}[hgrid, vgrid, x unit=5mm, time slot format=isodate]{2021-04-03}{2021-05-21}
		\gantttitlecalendar{week=8,month=shortname,day,weekday=letter} \\
		
		
		\ganttgroup{6}{2021-04-03}{2021-05-21} \\
		
		\ganttnewline
		
		%3.2 Verslag afwerken
		\ganttfoobar[name=32]{6.2}{2021-04-23}{2021-05-21} \\

		
		\ganttnewline
		
		%3.4 Presentatie afwerken
		\ganttfoobar[name=34]{6.4}{2021-04-23}{2021-05-21} \\
		
		
		%\ganttmilestone{Mijlpaal I}{2021-03-05} % milestone completed
		
	\end{ganttchart}
\end{figure}


\end{document}
