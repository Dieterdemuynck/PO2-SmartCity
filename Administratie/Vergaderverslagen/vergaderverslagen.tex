\documentclass[a4paper,twoside,kulak]{kulakreport} %options: kul or kulak (default)

%ik hoop dat het in dit bestandsstype moet en niet in kulakarticle

\usepackage[dutch]{babel}

\faculty{Wetenschap \& Technologie Kulak}
\group{Ingenieurswetenschappen}
\title{Vergaderverslagen}
\subtitle{Groep 6}
\author{Aaron Vandenberghe, Rani Jans, Sarah De Meester, Dieter Demuynck, Mathis Bossuyt, Jolien Barbier}
\institute{KU Leuven Kulak, Wetenschap \& Technologie}
\date{Academiejaar 2020 -- 2021}
\address{
   KU Leuven Kulak           \\
   Wetenschap \& Technologie \\
   Etienne Sabbelaan 53, 8500 Kortrijk             \\
   Tel.\ +32 56 24 60 20     \\
}

\begin{document} 

\titlepage

\tableofcontents
%De zinnen moeten vaak nog aangepast worden. Velen zijn wat lang of klopt de zinsbouw niet volledig dus dat zullen we moeten aanpassen. Dit is gewoon een ruwe versie
%Ik weet niet of er eigenlijk een table of contents moet zijn bij een vergaderverslag

\chapter*{Inleiding}
In dit document zijn de vergaderverslagen van groep 6 samengebundeld. Per sessie hebben we een begin- en eindvergadering
gehad. In de beginvergadering werd vooral besproken wat er die sessie gedaan moest worden en door wie.  Vaak splitsten we ons dan op in groepjes om elkaar te helpen binnen hetzelfde onderwerp. Bij de eindvergadering werd besproken of de doelen voor die dag behaald zijn en wat er al dan niet moet afgewerkt worden. 
%Heeft een vergaderverslag eigenlijk een inleiding?

\chapter{Sessie 1: 12 februari 2021}
% 'hoofdstuk ' moet misschien nog aangepast
\section{Begin sessie 1}
\subsection{Planning}
%teveel WE
Deze sessie  bestaat uit drie delen. Eerst krijgen we een inleiding om te weten wat er van ons verwacht wordt. Dan gaan we brainstormen over het project en daarna kunnen we effectief beginnen. We maken kennis met onze teamgenoten. We gaan de taakstructuur maken om zicht te krijgen op wat er moet gebeuren. Rani maakt een Onedrive aan om zo eenvoudig bestanden met elkaar te delen voor de deadline tegen dinsdag. Dieter zorgt voor de github omdat we volgende week daarmee leren werken en dit dus zeker nodig gaan hebben. %We hebben achteraf gezien dat de onedrive overbodig was maar toen konden we nog niet met GIT werken.
Sarah begint met de klantenvereisten. We gaan nadenken over de materialen die we mogelijks nodig hebben. We zijn bezig met de documenten die tegen dinsdag ingediend moeten worden. De takenstructuur hebben we ons eerst mee bezig gehouden om zo een planning op te stellen in de teamkalender. %Deze zinsbouw voelt raar.
Dieter doet opzoekingswerk over de onderdelen. We delen ons op in 2 groepen: een groepje voor de planning en een groepje voor het opzoekingswerk. In het groepje zelf hebben we de taken verdeeld, Jolien maakt de Ganttchart, Aaron de teamplanning en Rani de taakstructuur. 
\subsection{Groepsdynamiek}
We hebben de functies verdeeld en die in de verantwoordelijkheidsstructuur gezet. Aaron is de teamleider, Rani is de notulist, Jolien is de penningmeester, Dieter is softwareverantwoordelijke, Mathis is eindverantwoordelijke voor de constructie en Sarah is de planner. 
\section{Einde sessie 1}
\subsection{Evaluatie activiteiten en planning}
%Staat in vb apart maar weet niet wat het best is.
Rond vijf uur hadden we een vergadering gepland om samen te overleggen over de te bestellen materialen. Dieter en Mathis hebben al een idee gemaakt van welke materialen en sensoren nodig zijn. Ze hebben het uitgelegd zodat iedereen mee is. Dan zijn de onafgewerkte documenten verdeeld. Jolien werkt Gantt-chart af en Aaron zal de teamplanning in orde maken. Rani en Sarah verzamelen de afzonderlijke documenten in een pdf-bestand en dienen in. We zullen ook eens in het weekend samenkomen om opzoekingswerk te doen over de onderdelen.
\label{afwerking sessie 1}


\chapter{Sessie 2: 26 februari 2021}
\section{Begin sessie 2}
\subsection{Evaluatie activiteiten}
De zaken die af te werken waren zijn in orde gebracht. Daarvoor verwijs ik naar het vorige vergaderverslag (\ref{afwerking sessie 1}). De deadline is behaald. Ook hebben we zondag even samengekomen via discord en ons elk over een paar onderdelen verdiept en dan met elkaar deze informatie gedeeld. Zo leren we het best welke onderdelen samengaan voor ons ontwerp te kunnen maken.

\subsection{Rondvraag}
%Dit staat in de verleden tijd en bij de vorige sessie bij de beginplanning was tegenwoordige tijd. Ik twijfel wat best is.
We stellen tijdens onze vergadering twee alternatieven op voor onze auto. Op deze manier kan iedereen mee beslissen over ons wagentje. We hebben beide opties uitvoerig besproken zowel qua techniciteit en prijs. Daarna hebben we afgesproken wat ons maximaal bod is.
\subsection{Planning}
 Om 3 uur gaat Aaron naar de bieding. De rest zal zich ondertussen bezig houden met informatie opzoeken over hoe we het zouden maken. Sarah en Rani gaan in een groepje modelleren op Solid Edge. Dieter gaat kijken voor de implementatie. Aaron en Jolien houden zich bezig met de basislijst. Wanneer Aaron daarmee klaar is gaat hij Dieter helpen. Jolien zal dan mee modelleren. Mathis doet onderzoek naar de constructie en MyRio.

\section{Einde sessie 2}
\subsection{Evaluatie activiteiten}
 Sarah, Jolien en Rani zijn elk begonnen met modelleren van een onderdeel maar moeten het volgende sessie afwerken. Mathis heeft zich verdiept in de werking van MyRio. Aaron en Dieter zijn bezig geweest met Labview. 


\chapter{Sessie 3: 5 maart 2021}
\section{Begin sessie 3}
\subsection{Planning}
We hebben onze groepjes van vorige sessie behouden. Sarah, Jolien en Rani gaan verder met modelleren. Dieter en Mathis gaan werken in LabView. Aaron gaat bijspringen en aansturen waar nodig.  Jolien gaat de kleurensensor afgewerken. Rani gaat verder met de chassis. Sarah is bijna klaar met de ballcaster en het wiel dus gaat dit in deze sessie afwerken. Wanneer ze klaar zijn met deze onderdelen kijken ze naar de taakstructuur wat er nog moet gebeuren.

\section{Einde sessie 3}
\subsection{Evaluatie activiteiten}
Degenen die programmeerden en modelleerden hebben verteld hoe ver ze na vandaag staan. We hebben gevraagd of we een paar modellen van de producenten mogen gebruiken en dat mag, zolang we goed refereren. Aaron heeft verteld wat Martijn hem heeft gezegd over de administratieve documenten en het verslag. Er moeten nog wat zaken aangepast worden aan de administratieve documenten. Jolien heeft geholpen bij de ballcaster van Sarah en de gearmotor en de kleurensensor afgewerkt. Sarah heeft de ballcaster,wiel,motorbeugel af. Rani is klaar met de chassis. Ze heeft maar een onderdeel kunnen doen door wat problemen. Ze moest twee keer opnieuw beginnen met het chassis. De eerste keer had Solid Edge de afmetingen aangepast. De tweede keer viel haar VPN-verbinding weg en sloot Solid Edge opeens af waardoor de wijzigingen verloren gingen aan de nieuwe chassis. Uiteindelijk is het toch gelukt. Dieter heeft ons wat meer uitleg gegeven over Github en Gitkraken. Het team van de implementatie is bezig geweest met research over LabView en MyRio.

\subsection{Planning}
Aaron gaat de feedback van Martijn gebruiken en het document aanpassen. Dieter is van plan om een kanban via Gitkrakenboards te maken zodat het duidelijker is om te zien wie met wat bezig is en dubbel werk vermeden kan worden.
 

\chapter{Sessie 4: 12 maart 2021}
\section{Begin sessie 4}
\subsection{Evaluatie activiteiten}
Dieter heeft een extra discord server gemaakt omdat we enkel zo telkens meldingen kunnen krijgen als er iets gepushed en gepulled is op Gitkraken. Ook heeft hij een soort kanban via Gitkrakenboards gemaakt. Aaron heeft de opmerkingen van Martijn aangepast.
\subsection{Planning}
Rani gaat zich bezig houden met wat er precies in het verslag moet en ook het vergaderverslag in \LaTeX te zetten met subsections zoals in het voorbeeld. Mathis gaat de link voor de CAD-modellen doorsturen want hij heeft modellen van de producent gevonden. Dan sluit hij weer aan bij Dieter en Aaron om opzoekingswerk te doen omtrent de implementatie. Ze willen nu vooral te weten komen wat met elkaar verbonden is en nog wat opzoeking over LabView. Sarah en Jolien gaan weer modelleren en technische tekening maken. Ze zijn van plan om vandaag alle technische tekeningen en modellen af te werken.De onderdelen die nog moeten gebeuren zijn de kleurensensor, de motorbeugel, de gearmotor, de chassis, het wiel, de ballcaster, de dualdrivemotor, de microcontroller, de afstandssensor en de reflexiesensor.
\section{Einde sessie 4}
\subsection{Evaluatie activiteiten}
Jolien en Sarah hebben de technische tekeningen van de kleurensensor, de motorbeugel, de gearmotor, de chassis, het wiel en de ballcaster gemaakt.Jolien is nog bezig met de afstandssensor. Rani heeft de vergaderverslagen in \LaTeX gezet en de doorlopende teksten van de eerdere verslagen mooi opgesplitst in de juiste subtitles zoals het voorbeeld op toledo. Ze is bij de groep van modelleren en implementeren gaan vragen waar ze nu exact zitten om al een globaal zicht te krijgen op waar we nu zitten en wat er afgewerkt moet worden. Ze heeft de stuklijst in LaTeX gezet. Dieter, Aaron en Mathis hebben opnieuw opzoeking gedaan naar LabView. Dieter heeft zich specifieker gericht op het oplossen van problemen met LabView en Gitkraken om zo het team te helpen. Aaron heeft al wat basisimplementatie geschreven en wat uitgetest. Het is gelukt om de motor te doen draaien via het programma. Hij zoekt ook naar welke onderdelen samen kunnen werken. Mathis heeft zich ook bezig gehouden met de 3D modellen. Hij heeft de reflexiesensor gemodelleerd en Sarah geholpen met Solid Edge.

\subsection{Planning}
Mathis gaat in het weekend nog wat oefenen op LabView.  
Sarah wil nog een of twee technische tekeningen afwerken.

\subsection{Planning volgende sessie}
We spreken nu al af wat we volgende week gaan doen omdat het dan op campus is en dus moeilijker om in groep te communiceren. Rani en Jolien gaan werken aan het eindverslag. Sarah gaat de 3D- assemblage doen en de technische tekeningen afwerken. Als ze daarmee klaar is zal ze meewerken de implementatie. Mathis, Aaron en Dieter gaan weer verder met verkennen. Bijvoorbeeld hoe alles werkt en samengaat van programma's en of het op zichzelf werkt. %deze zinnen moeten nog aangepast.
\label{Planning sessie5}

\chapter{Sessie 5: 19 maart 2021}
\section{Begin sessie 5}
De planning is bij het einde van de vorige sessie al bepaald. Daarvoor verwijs ik dus naar het vorige vergaderverslag (\ref{Planning sessie5}).

\section{Einde sessie 5}


\chapter{Sessie 6: 26 maart 2021}
\section{Begin sessie 6}

\section{Einde sessie 6}


\chapter{Sessie 7: 2 april 2021}
\section{Begin sessie 7}

\section{Einde sessie 7}












\chapter*{Besluit}
Afsluitende tekst.

\end{document}
