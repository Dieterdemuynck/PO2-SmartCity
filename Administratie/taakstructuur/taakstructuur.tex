\documentclass[12pt]{article}
\usepackage[dutch]{babel}
% style
\setlength{\parindent}{0pt}
\usepackage[left=2.5cm,top=2cm,right=2.5cm,bottom=2cm,a4paper]{geometry}
\usepackage{fancyhdr}
\pagestyle{fancy}
\lhead{Team 6}
\rhead{Taakstructuur}
\renewcommand{\headrulewidth}{0.4pt}

% table
\usepackage{etoolbox}
\usepackage{booktabs}
\usepackage{xstring}
\usepackage{xcolor,colortbl}
\definecolor{gray1}{gray}{0.7}
\definecolor{gray2}{gray}{0.85}
\definecolor{gray3}{gray}{0.95}

% counting system		
\newcounter{counter}
\newcounter{subcounter}
\newcounter{subsubcounter}
\newcommand{\teller}{
	\stepcounter{counter}
	\setcounter{subcounter}{0}
	\setcounter{subsubcounter}{0}
	\thecounter}
\newcommand{\subteller}{
	\stepcounter{subcounter}
	\setcounter{subsubcounter}{0}
	\thecounter.\thesubcounter}
\newcommand{\subsubteller}{
	\stepcounter{subsubcounter}
	\thecounter.\thesubcounter.\thesubsubcounter}

% row	
\newcommand{\row}[3]{
	\IfEqCase{#1}{
		{1}{\teller & #2 & \IfEqCase{#3}{{0}{niet OK}{1}{OK}} \\}
		{2}{\subteller & #2 & \IfEqCase{#3}{{0}{niet OK}{1}{OK}} \\}
		{3}{\subsubteller & #2 & \IfEqCase{#3}{{0}{niet OK}{1}{OK}} \\}
	}[PackageError{row}{Undefined option to row: #1}]}



\begin{document}	
	
	\begin{table}
		\centering
		\begin{tabular}{p{1cm}p{12cm}c}
			\toprule
			Code & Taak & Status \\ 
			\midrule
			\row{1}{Inwerken}{1}
			\row{2}{Documenten op Toledo lezen}{1}
			\row{2}{Brainstormen}{1}
			\row{3}{Materiaal en onderdelen bespreken}{1}
			\row{3}{Individuele touch kiezen}{1}
			\row{3}{Klantenvereisten}{1}
			\row{2}{Plannen}{1}
			\row{3}{Planning op lange termijn (Gantt-chart)}{1}
			\row{3}{Teamkalender}{1}
			\row{3}{Taakstructuur}{1}
			\midrule
			%Verdere stappen nog aan te vullen
			\row{1}{Technische aspecten}{0}
			\row{2}{Stuklijst}{0}
			\row{3}{Lijst maken}{0}
			\row{3}{Onderdelen bestellen}{0}
			\row{2}{Assemblage}{0}
			\midrule
			\row{1}{3D modellen	(\emph{Solid parts})}{0}
			%grotere structuren
			\row{2}{Wiel}{1}
			\row{2}{Gearmotor}{1}
			\row{2}{Kleursensor}{1}
			\row{2}{Afstandssensor}{1}
			\row{2}{Reflectiesensor}{0}
			\row{2}{Chassis}{1}
			\row{2}{Frame}{0}
			\row{2}{Microcontroller}{0}
			\row{2}{Dual drive motor}{0} %nog andere onderdelen toevoegen
			\row{2}{Motorbeugel}{1}
			\row{2}{Balcaster}{1}
			\midrule
			\row{1}{Technische tekeningen	(\emph{Drawing})}{0}
			\row{2}{Wiel}{0}
			\row{2}{Gearmotor}{0}
			\row{2}{Kleursensor}{0}
			\row{2}{Afstandssensor}{0}
			\row{2}{Reflectiesensor}{0}
			\row{2}{Chassis}{1}
			\row{2}{Frame}{0}
			\row{2}{Microcontroller}{0}
			\row{2}{Dual drive motor}{0} %nog andere onderdelen toevoegen
			\row{2}{Motorbeugel}{0}
			\row{2}{Balcaster}{0} %nog andere onderdelen toevoegen
			\midrule
			\row{1}{Implementatie}{0}
			\row{2}{Straten volgen}{0}
			\row{2}{Wagens detecteren}{0}
			\row{2}{Verkeerslichten interpreteren}{0}
			\row{2}{Bestuurbaar vanop afstand}{0}
			\row{2}{Redelijke snelheid}{0}
			\row{2}{Wagentje testen}{0}
			\midrule
		\end{tabular}
	\end{table}
	\begin{table}
		\begin{tabular}{p{1cm}p{12cm}c}
			\toprule
			Code & Taak & Status \\ 
			\midrule
			\row{1}{Rapportering}{0}
			\row{2}{Tussentijds verslag maken}{0}
			\row{3}{Inleiding}{0}
			\row{3}{Ontwerpproces}{0}
			\row{3}{Planning}{0}
			\row{3}{Verslag maken}{0}
			\row{3}{Conclusie}{0}
			\row{3}{Nalezen}{0}
			\row{2}{Verslag afwerken}{0}
			\row{3}{Feedback in rekening brengen}{0}
			\row{2}{Tussentijdse presentatie}{0}
			\row{3}{Structuur}{0}
			\row{3}{Presentatie maken}{0}
			\row{3}{Nalezen}{0}
			\row{3}{Inoefenen}{0}
			\row{2}{Presentatie afwerken}{0}
			\row{3}{Feedback in rekening brengen}{0}
			\bottomrule
		\end{tabular}
	\end{table}
	
	Opmerking: Dit zal nog worden geüpdatet tijdens het semester.

\end{document}
