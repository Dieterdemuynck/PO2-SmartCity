\documentclass[a4paper,twoside,kulak]{kulakreport} %options: kul or kulak (default)

%ik hoop dat het in dit bestandsstype moet en niet in kulakarticle

\usepackage[dutch]{babel}

\faculty{Wetenschap \& Technologie Kulak}
\group{Ingenieurswetenschappen}
\title{Vergaderverslagen}
\subtitle{Groep 6}
\author{Aaron Vandenberghe, Rani Jans, Sarah De Meester, Dieter Demuynck, Mathis Bossuyt, Jolien Barbier}
\institute{KU Leuven Kulak, Wetenschap \& Technologie}
\date{Academiejaar 2020 -- 2021}
\address{
   KU Leuven Kulak           \\
   Wetenschap \& Technologie \\
   Etienne Sabbelaan 53, 8500 Kortrijk             \\
   Tel.\ +32 56 24 60 20     \\
}

\begin{document} 

\titlepage

\tableofcontents

%Ik weet niet of er eigenlijk een table of contents moet zijn bij een vergaderverslag

\chapter*{Inleiding}
In dit document zijn de vergaderverslagen van groep 6 samengebundeld. Per sessie hebben we een begin- en eindvergadering
gehad. In de beginvergadering werd besproken wat er die sessie gedaan moest worden en door wie. Vaak splitsten we ons dan op in groepjes om elkaar te helpen binnen hetzelfde onderwerp. Bij de eindvergadering werd besproken of de doelen voor die dag behaald zijn en wat er al dan niet moet afgewerkt worden. 
%Heeft een vergaderverslag eigenlijk een inleiding?

\chapter{Sessie 1: 12 februari 2021}
% 'hoofdstuk ' moet nog aangepast
\section{Begin sessie 1}
\subsection{Planning}
Deze sessie  bestaat uit drie delen. Eerst krijgen we een inleiding om te weten wat er van ons verwacht wordt. Dan gaan we brainstormen over het project en daarna kunnen we effectief beginnen. We maken kennis met onze teamgenoten. We gaan de taakstructuur maken om zicht te krijgen op wat er moet gebeuren. Rani maakt een Onedrive aan om zo eenvoudig bestanden met elkaar te delen voor de deadline tegen dinsdag. Dieter zorgt voor de github omdat we volgende week daarmee leren werken en dit dus zeker nodig gaan hebben. %We hebben achteraf gezien dat de onedrive overbodig was maar toen konden we nog niet met GIT werken.
Sarah begint met de klantenvereisten. We gaan brainstormen over de materialen die we mogelijks nodig hebben. We zijn bezig met de documenten die tegen dinsdag ingediend moeten worden. De takenstructuur hebben we ons eerst mee bezig gehouden om zo een planning op te stellen in de teamkalender. %Deze zinsbouw voelt raar.
Dieter doet opzoekingswerk over de onderdelen. We delen ons op in 2 groepen: een groepje voor de planning en een groepje voor het opzoekingswerk. In het groepje zelf hebben we de taken verdeeld, Jolien maakt de Ganttchart, Aaron de teamplanning en Rani de taakstructuur. 
\subsection{Groepsdynamiek}
We hebben de functies verdeeld en die in de verantwoordelijkheidsstructuur gezet. Aaron is de teamleider, Rani is de notulist, Jolien is de penningmeester, Dieter is softwareverantwoordelijke, Mathis is eindverantwoordelijke voor de constructie en Sarah is de planner. 
\section{Einde sessie 1}
\subsection{Evaluatie activiteiten}
Rond vijf uur hadden we een vergadering gepland om samen te overleggen over de te bestellen materialen. Dieter en Mathis hebben al een idee gemaakt van welke materialen en sensoren nodig zijn. Ze hebben het uitgelegd zodat iedereen mee is. Dan zijn de onafgewerkte documenten verdeeld. Jolien werkt Gantt-chart af en Aaron zal de teamplanning in orde maken. Rani en Sarah verzamelen de afzonderlijke documenten in een pdf-bestand en dienen in.



\chapter{Sessie 2: 26 februari 2021}
\section{Begin sessie 2}
We hebben 2 alternatieven opgesteld voor onze auto. We hebben dit uitvoerig besproken zowel qua techniciteit en prijs. Dan hebben we afgesproken wat ons maximaal bod is. Om 3 uur is Aaron naar de bieding gegaan. We hebben ons als team ondertussen bezig gehouden met info opzoeken over hoe we het zouden maken. We kiezen voor MyRio. Sarah en Rani gaan in een groepje modelleren op Solid Edge. Dieter gaat kijken voor de implementatie. Aaron en Jolien houden zich bezig met de basislijst. Wanneer Aaron daarmee klaar is gaat hij Dieter helpen. Mathis doet onderzoek naar de constructie en MyRio.
\section{Einde sessie 2}
De week die komt is al enorm druk dus zullen we het hierbij laten. Sarah, Jolien en Rani zijn elk begonnen met modelleren van een onderdeel. Mathis heeft zich verdiept in de werking van MyRio. Aaron en Dieter zijn bezig geweest met Labview.

\chapter{Sessie 3: 5 maart 2021}
\section{Begin sessie 3}
We hebben onze groepjes van vorige sessie behouden. Sarah, Jolien en Rani gaan modelleren. Dieter en Mathis gaan werken in LabView. Aaron gaat bijspringen en aansturen waar nodig. Dieter heeft ons wat meer uitleg gegeven over Github en Gitkraken. Jolien heeft de kleurensensor afgewerkt. Sarah is bijna klaar met de ballcaster en het wiel. Rani is klaar met de chassis.
\section{Einde sessie 3}
Degenen die programmeerden en modelleerden hebben verteld hoe ver ze na vandaag staan. We hebben gevraagd of we een paar modellen van de producenten mogen gebruiken en dat mag, zolang we goed refereren.  Aaron heeft verteld wat Martijn hem heeft gezegd over de administratieve documenten en het verslag. Hij gaat de laatste aanpassingen vandaag of morgen doen. Jolien heeft de ballcaster en gearmotor afgewerkt. Sarah heeft de motorbeugel af. 

\chapter{Sessie 4: 12 maart 2021}
\section{Begin sessie 4}
Dieter heeft een extra discord server gemaakt omdat we enkel zo telkens meldingen kunnen krijgen als er iets gepushed en gepulled is. Ook heeft hij een soort kanban via Gitkrakenboards gemaakt zodat het duidelijk blijft wie met wat bezig is. Aaron heeft de opmerkingen van Martijn aangepast. Rani gaat zich bezig houden met wat er in het verslag moet en ook het vergaderverslag in LaTeX te zetten. Mathis gaat de link voor de cadmodellen doorsturen want hij heeft modellen van de producent gevonden. Dan sluit hij weer aan bij Dieter om te implementeren. Sarah en Jolien gaan weer modelleren en technische tekening maken. Aaron gaat ook meehelpen voor de implementatie. Ze gaan beginnen met het manuele besturen. 
\section{Einde sessie 4}


\chapter{Sessie 5: 19 maart 2021}
\section{Begin sessie 5}

\section{Einde sessie 5}


\chapter{Sessie 6: 26 maart 2021}
\section{Begin sessie 6}

\section{Einde sessie 6}


\chapter{Sessie 7: 2 april 2021}
\section{Begin sessie 7}

\section{Einde sessie 7}












\chapter*{Besluit}
Afsluitende tekst.

\end{document}
