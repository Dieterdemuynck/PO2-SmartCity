\documentclass[a4paper,twoside,kulak]{kulakreport} %options: kul or kulak (default)

%ik hoop dat het in dit bestandsstype moet en niet in kulakarticle

\usepackage[dutch]{babel}

\faculty{Wetenschap \& Technologie Kulak}
\group{Ingenieurswetenschappen}
\title{Vergaderverslagen}
\subtitle{Groep 6}
\author{Aaron Vandenberghe, Rani Jans, Sarah De Meester, Dieter Demuynck, Mathis Bossuyt, Jolien Barbier}
\institute{KU Leuven Kulak, Wetenschap \& Technologie}
\date{Academiejaar 2020 -- 2021}
\address{
   KU Leuven Kulak           \\
   Wetenschap \& Technologie \\
   Etienne Sabbelaan 53, 8500 Kortrijk             \\
   Tel.\ +32 56 24 60 20     \\
}

\begin{document} 

\titlepage

\tableofcontents

%Ik weet niet of er eigenlijk een table of contents moet zijn bij een vergaderverslag

\chapter*{Inleiding}
In dit document zijn de vergaderverslagen van groep 6 samengebundeld. Per sessie hebben we een begin- en eindvergadering
gehad. In de beginvergadering werd besproken wat er die sessie gedaan moest worden en door wie. Vaak splitsten we ons dan op in groepjes om elkaar te helpen binnen hetzelfde onderwerp. Bij de eindvergadering werd besproken of de doelen voor die dag behaald zijn en wat er al dan niet moet afgewerkt worden. 
%Heeft een vergaderverslag eigenlijk een inleiding?

\chapter{Sessie 1: 12 februari 2021}
% 'hoofdstuk ' moet nog aangepast
\section{Begin sessie 1}

\section{Einde sessie 1}


\chapter{Sessie 2: 26 februari 2021}
\section{Begin sessie 2}

\section{Einde sessie 2}


\chapter{Sessie 3: 5 maart 2021}
\section{Begin sessie 3}

\section{Einde sessie 3}


\chapter{Sessie 4: 12 maart 2021}
\section{Begin sessie 4}

\section{Einde sessie 4}


\chapter{Sessie 5: 19 maart 2021}
\section{Begin sessie 5}

\section{Einde sessie 5}











\chapter*{Besluit}
Afsluitende tekst.

\end{document}
